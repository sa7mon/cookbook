\recipe{BBQ Chicken Drumsticks}
\servingstimes{3}{10 min}{1hr 10 min}
\category{Dinners}

\ingredients
\begin{multicols}{2}
\raggedcolumns % Don't make the second column take the vertical space of the first column
\begin{itemize}
    \item 6 chicken drumsticks
    \item \nicefrac{1}{2} cup water
    \item \nicefrac{1}{2} cup ketchup
    \item \nicefrac{1}{3} cup white vinegar
    \item \nicefrac{1}{4} cup brown sugar
    \item 4 teaspoons butter
    \item 2 teaspoons salt
    \item 2 teaspoons Worcestershire sauce
    \item 2 teaspoons dry mustard
    \item 2 teaspoons chili powder
\end{itemize}
\end{multicols}

\instructions
\begin{enumerate}
    \item Preheat oven to \temp{400}. Place drumsticks in a baking dish.
    \item Whisk water, ketchup, vinegar, brown sugar, butter, salt, Worchestershire sauce, mustard, and chili powder together in a bowl; pour mixture over drumsticks. Cover with aluminum foil. 
    \item Baker in preheated oven until no longer pink at the bone and the juices run clear, about 1 hour, turning chicken about halfway through. An instant-read thermometer inserted near the bone should read \temp{165}.
\end{enumerate}

\vfill

\begin{tabular}{ c | l | l }
  \textbf{Version} & \textbf{Date} & \textbf{Changes} \\ 
  \hline		
  1.0 & 2020-08-28 & Original \\
\end{tabular}

%%%%%%%%%%%%%%%%%%%%%%%%%%%%%%%%%%%%%%%%%%%%%%%%%%%%%%%%%%%%%%%%%%%%%%%%%%%%%

\recipe{Chicken Salad}
\servingstimes{3}{25 min}{0 min}
\category{Dinners}

\ingredients
\begin{multicols}{2}
\raggedcolumns % Don't make the second column take the vertical space of the first column
\begin{itemize}
    \item 2 cups cooked, shredded chicken 
    \item \nicefrac{1}{2} cup chopped celery
    \item 1 tablespoon chopped parsley leaves
    \item \nicefrac{1}{2} teaspoon coarse salt
    \item \nicefrac{1}{4} teaspoon fresh ground black pepper
    \item 1 tablespoon fresh lemon juice
    \item \nicefrac{1}{3} cup mayonnaise
    \item 3 teaspoons unsalted butter (softened)
    \item 6 romaine lettuce leaves (cleaned, patted dry)
\end{itemize}
\end{multicols}

\instructions
\begin{enumerate}
    \item Combine in a bowl and serve on bread
\end{enumerate}

\vfill

\begin{tabular}{ c | l | l }
  \textbf{Version} & \textbf{Date} & \textbf{Changes} \\ 
  \hline		
  1.0 & 2020-08-27 & Original \\
\end{tabular}


%%%%%%%%%%%%%%%%%%%%%%%%%%%%%%%%%%%%%%%%%%%%%%%%%%%%%%%%%%%%%%%%%%%%%%%%%%%%%

\recipe{Italian Meatballs}
\servingstimes{8}{20 min}{10 min}
\category{Dinners}

\ingredients
\begin{multicols}{2}
\raggedcolumns % Don't make the second column take the vertical space of the first column
\begin{itemize}
    \item 1lb ground beef or turkey, thawed
    \item 3 cloves garlic
    \item \nicefrac{1}{2} tsp onion powder
    \item \nicefrac{1}{4} tsp basil
    \item \nicefrac{1}{4} tsp oregano
    \item \nicefrac{1}{2} tsp rosemary
    \item 8 saltine crackers
    \item 1 egg
\end{itemize}
\end{multicols}

\instructions
\begin{enumerate}
    \item Pre-heat oven to \temp{350}.
    \item Crush saltines and add to bowl. Add rest of ingredients to bowl.
    \item Shape into balls and place on baking sheet.
    \item Bake for 20 minutes.
\end{enumerate}

\vfill

\begin{tabular}{ c | l | l }
  \textbf{Version} & \textbf{Date} & \textbf{Changes} \\ 
  \hline		
  1.0 & 2022-08-23 & Original \\
\end{tabular}

%%%%%%%%%%%%%%%%%%%%%%%%%%%%%%%%%%%%%%%%%%%%%%%%%%%%%%%%%%%%%%%%%%%%%%%%%%%%%

\recipe{Korean Beef and Rice Bowls}
\servingstimes{3}{5 min}{30 min}
\category{Dinners}

\ingredients
\begin{multicols}{2}
\raggedcolumns % Don't make the second column take the vertical space of the first column
\begin{itemize}
    \item 1 lb ground beef (80/20)
    \item 3 garlic cloves, minced
    \item \nicefrac{1}{4} cup packed brown sugar
    \item \nicefrac{1}{4} cup low-sodium soy sauce
    \item 1 teaspoons sesame oil
    \item \nicefrac{1}{4} teaspoon ground ginger
    \item \nicefrac{1}{4} teaspoon crushed red pepper flakes
    \item \nicefrac{1}{4} teaspoon pepper
    \item \nicefrac{1}{2} cup diced carrots \& celery
\end{itemize}
\end{multicols}

\instructions
\begin{enumerate}
    \item In a large skillet cook the ground beef and garlic breaking it into crumbles over medium heat. Saute vegetables.
    \item In a small bowl, whisk brown sugar, soy sauce, sesame oil, ginger, red pepper flakes, and pepper. Pour over the ground beef and let simmer for another minute or two.
    \item Server over hot white rice and garnish with sesame seeds.
\end{enumerate}

\vfill

\begin{tabular}{ c | l | l }
  \textbf{Version} & \textbf{Date} & \textbf{Changes} \\ 
  \hline		
  1.2 & 2020-09-10 & Reduce red pepper \& add vegetables \\
  1.1 & 2020-08-04 & Reduce sesame oil \\
  1.0 & 2020-04-01 & Original \\
\end{tabular}

%%%%%%%%%%%%%%%%%%%%%%%%%%%%%%%%%%%%%%%%%%%%%%%%%%%%%%%%%%%%%%%%%%%%%%%%%%%%%

\recipe{Maple \& Sage Pork Chops}
\servingstimes{4}{15 min}{10 min}
\category{Dinners}

\ingredients
\begin{multicols}{2}
\raggedcolumns % Don't make the second column take the vertical space of the first column
\begin{itemize}
    \item 2 tbsp finely chopped fresh sage
    \item 2 tsp olive oil
    \item \nicefrac{1}{4} tsp salt
    \item 4 boneless pork chops (about 4oz each)
    \item 2 tsp maple syrup
\end{itemize}
\end{multicols}

\instructions
\begin{enumerate}
    \item Combine 2 tbsp safe, oil, and salt in small bowl. Rub mixture evenly over pork chops. Place on rimmed baking sheet.
    \item Broil pork chops 4 minutes. Turn over, brush evenly with syrup. Broil 4 minutes or until pork chops are browned and barely pink in center. Garnish with additional sage and serve with applesauce.
\end{enumerate}

\vfill

% nutrition per serving

% Calories 203
% Protein 25g
% Carbs 2g
% Total Fat: 10g
% Saturated Fat: 3g
% Cholesterol: 62mg


\begin{tabular}{ c | l | l }
  \textbf{Version} & \textbf{Date} & \textbf{Changes} \\ 
  \hline		
  1.0 & 2020-04-01 & Original from 'Winner! Winner! 300 Calorie Dinners' \\
\end{tabular}

%%%%%%%%%%%%%%%%%%%%%%%%%%%%%%%%%%%%%%%%%%%%%%%%%%%%%%%%%%%%%%%%%

\recipe{Orange Chicken}
\servingstimes{6}{20 min}{15 min}
\category{Dinners}

% \ingredients


\begin{multicols}{2}
\textbf{Chicken}
\raggedcolumns 
\begin{itemize}
    \item 2lb chicken breast/tenderloin
    \item \nicefrac{1}{2} tsp salt
    \item \nicefrac{1}{2} tsp ground pepper
    \item \nicefrac{1}{2} cup cornstarch
    \item \nicefrac{1}{4} cooking oil
    
    \columnbreak
    
    \textbf{Orange Sauce}
    
    \item \nicefrac{1}{4} cup soy sauce
    \item \nicefrac{1}{4} cup rice vinegar
    \item \nicefrac{1}{4} cup orange juice
    \item \nicefrac{1}{4} cup water
    \item \nicefrac{1}{4} cup brown sugar, packed
    \item 1 tbsp cornstarch
    \item \nicefrac{1}{2} tsp minced garlic
    \item \nicefrac{1}{4} tsp crushed red pepper flakes
    \item 1 tbsp orange zest
    \item green onions, to garnish
\end{itemize}
\end{multicols}

\instructions
\begin{enumerate}
    \item Cut chicken into bite-sized pieces and heat a heavy-bottomed skillet over medium-high heat
    \item Place the cornstarch in a large bowl and season with salt and pepper, add the chicken on top and toss until evenly coated.
    \item Add just enough oil to coat the bottom of the hot skillet and then add the chicken.
    \item Cook stirring occasionally until all the chicken is cooked through and golden brown.
    \item Meanwhile, whisk together all the sauce ingredients in a small bowl.
    \item Once the chicken is cooked through, add the sauce to the hot skillet and simmer and stir until thickened.
    \item Garnish with some green onions if you want, and enjoy. You can serve it with steamed rice if desired.

    
\end{enumerate}

\vfill

\begin{tabular}{ c | l | l }
  \textbf{Version} & \textbf{Date} & \textbf{Changes} \\ 
  \hline		
  1.0 & 2022-08-23 & Original \\
\end{tabular}

%%%%%%%%%%%%%%%%%%%%%%%%%%%%%%%%%%%%%%%%%%%%%%%%%%%%%%%%%%%%%%%%%%

\recipe{Pad Thai}
\servingstimes{4}{15 min}{15 min}
\category{Dinners}

\ingredients
\begin{multicols}{2}
\raggedcolumns 
\begin{itemize}
    \item 8oz flat rice noodles
    \item 3 tablespoons oil
    \item 3 cloves garlic, minced
    \item 8oz uncooked shrimp, chicken, or tofu
    \item 2 eggs
    \item 1 cup fresh bean sprouts
    \item 1 red bell pepper, thinly sliced
    \item 3 green onions, chopped
\columnbreak
    \item \nicefrac{1}{2} cup dry roasted peanuts
    \item 2 limes
    \item \nicefrac{1}{2} fresh cilantro, chopped
\end{itemize}


\textbf{Pad Thai Sauce}
\begin{itemize}
    \item 3 tablespoons fish sauce
    \item 1 tablespoon low-sodium soy sauce
    \item 5 tablespoons light brown sugar
    \item 2 tablespoons rice vinegar
    \item 2 tablespoons creamy peanut butter, optional
\end{itemize}
\end{multicols}
\instructions
\begin{enumerate}
    \item Cook noodles according to package instructions, just until tender. Rinse under cold water.
    \item Mix the sauce ingredients together and set aside.
    \item Heat 1\nicefrac{1}{2} tablespoons of oil in a large saucepan or wok over medium-high heat.
    \item Add the shrimp/chicken/tofu, garlic, and bell pepper. The shrimp will cook quickly, about 1-2 minutes on each side, or until pink. If using chicken, cook until just cooked through, about 3-4 minutes, flipping only once.
    \item Push everything to the side of the pan. Add a little more oil and add the beaten eggs. Scramble the eggs, breaking them into small pieces with a spatula as they cook.
    \item Add noodles, sauce, bean sprouts, and peanuts to the pain (reserving some peanuts for topping). Toss everything to combine.
    \item Top with green onions, extra peanuts, cilantro, and lime wedges. Serve immediately!
\end{enumerate}

\vfill

\begin{tabular}{ c | l | l }
  \textbf{Version} & \textbf{Date} & \textbf{Changes} \\ 
  \hline		
  1.0 & 2020-08-29 & Original \\
\end{tabular}

%%%%%%%%%%%%%%%%%%%%%%%%%%%%%%%%%%%%%%%%%%%%%%%%%%%%%%%%%%%%%%%%%%%%%%%%%%%%%

\recipe{Slow Cooker Pulled Pork}
\servingstimes{8}{15 min}{5 hours}
\category{Dinners}

\ingredients
\begin{multicols}{2}
\raggedcolumns 
\begin{itemize}
    \item 1 teaspoon vegetable oil
    \item 1 (4lb) pork shoulder roast
    \item 1 cup barbeque sauce
    \item \nicefrac{1}{2} cup apple cider vinegar
    \item \nicefrac{1}{2} cup chicken broth
    \item \nicefrac{1}{4} cup light brown sugar
    \item 1 tablespoon prepared yellow mustard
    \item 1 tablespoon Worcestershire sauce
    \item 1 tablespoon chili powder
    \item 1 extra large onion, chopped
    \item 2 large cloves garlic, chopped
    \item 1 \nicefrac{1}{2} teaspoons dried thyme
    \item 2 tablespoons butter
    \item 8 hamburger buns, split
    \item 2 bay leaves
\end{itemize}
\end{multicols}

\instructions
\begin{enumerate}
    \item Pour the vegetable oil into the bottom of the slow cooker. Place the port roast into the slow cooker; pour in the barbecue sauce, apple cider vinegar, and chicken broth. 
    \item Stir in the brown sugar, yellow mustard, Worcestershire sauce, chili powder, onion, garlic, and thyme.
    \item Cover and cook  on High until the roast shreds easily with a fork, 5 to 6 hours.
    \item Remove the roast from the slow cooker, and shred the meat using two forks. Return the shredded pork to the slow cooker, and stir the meat into the juices. 
\end{enumerate}

\vfill

\begin{tabular}{ c | l | l }
  \textbf{Version} & \textbf{Date} & \textbf{Changes} \\ 
  \hline		
  1.1 & 2020-08-28 & Add bay leaves \\
  1.0 & 2020-07-08 & Original \\
\end{tabular}

%%%%%%%%%%%%%%%%%%%%%%%%%%%%%%%%%%%%%%%%%%%%%%%%%%%%%%%%%%%%%%%%%%%%%%%%%%%%%
\recipe{Stuffed Peppers}
\servingstimes{4}{10 min}{30 min}
\category{Dinners}

\ingredients
\begin{multicols}{2}
\raggedcolumns % Don't make the second column take the vertical space of the first column
\begin{itemize}
    \item 1 lb ground beef (80/20)
    \item 1 can Rotel tomatoes
    \item 4 green bell peppers
    \item \nicefrac{1}{4} shredded cheese
    \item 1 packet taco seasoning
    \item 1 cup Mexican rice, cooked
\end{itemize}
\end{multicols}

\instructions
\begin{enumerate}
    \item In a large skillet cook the ground beef over medium heat. Add taco seasoning.
    \item Combine cooked rice into beef.
    \item Clean and hollow out peppers. Place on rimmed cooking sheet.
    \item Add cheese to the bottom of each pepper.
    \item Fill each pepper with the rice and meat mixture, then top with cheese.
    \item Cover with tin foil and bake for 30 minutes.
    \item Remove foil and continue baking until cheese is browned.
\end{enumerate}

\vfill

\begin{tabular}{ c | l | l }
  \textbf{Version} & \textbf{Date} & \textbf{Changes} \\ 
  \hline		
  1.0 & 2020-04-01 & Original from Jones \\
\end{tabular}
